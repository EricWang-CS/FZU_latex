% !TeX document-id = {e2d5d229-2be4-4fd9-b208-1188e0212a37}
\documentclass[master]{thesis}
\usepackage{amsfonts}
\usepackage{amssymb}
\usepackage{amsthm,amsmath}
\usepackage{mathrsfs}
\usepackage{indentfirst}
\usepackage{multirow}
\usepackage{threeparttable}
\usepackage{subcaption}
%\usepackage{algorithm}%
%\usepackage{algorithmicx}%
% 定义本论文相关元信息
\theoremstyle{definition} % 样式为定义
\newcommand{\upcite}[1]{\textsuperscript{\textsuperscript{\cite{#1}}}}
\def\thesisChineseBookName{基于xxx算法研究}
\def\thesisEnglishBookName{Research on xxx}

\title{\thesisChineseBookName}{\thesisEnglishBookName}


\SetKwInOut{KIN}{输入}
\SetKwInOut{KOUT}{输出}

\begin{document}
% 插入封面
\thesisTitlePage
% 插入声明页
% \thesisDeclarationPage

% 中文摘要及关键词
\begin{chineseAbstract}

\chinesekeyword{分类;}
%\let
%\cleardoublepage
%\clearpage
\end{chineseAbstract}

% 英文摘要及关键词
\begin{englishAbstract}

\end{englishAbstract}
\let\cleardoublepage\clearpage
%% 目录
\thesisContents
%% 主要符号对照表
%\begin{thesisMainSymbol}
%    $\mathbb{R}^d$        & $d$维Euclidean空间        \\
%    $n$                    & 输入样本个数               \\
%    $\mathbf{X}$           & 输入特征矩阵               \\
%    $\mathbf{x}_i$         & 第$i$个样本的特征向量      \\
%\end{thesisMainSymbol}
%
%%% 缩略词表
%\begin{thesisAcronyms}
%    GNNs & Graph Neural Networks         & 图神经网络    \\
%    GCN  & Graph Convolution Network     & 图卷积网络    \\
%\end{thesisAcronyms}
%############第一章##############
\chapter{绪论}
\let\cleardoublepage\clearpage
\label{chapter_introduction}
\renewcommand{\headrulewidth}{0.4pt} % 在目录、符号表、缩略词表中去掉页眉下划线,这边需要恢复
%图片用\ref{XXX}引用
% \begin{figure}[!t]
%     \centering
%     \includegraphics[width=16.5cm]{XXX.pdf}
%     \caption{XXXXXX}
%     \label{chapter_sgae_fig_overview}
% \end{figure}


%############第二章##############
\chapter{相关定义与工作}


%############第三章##############
\chapter{第三章}

%############第四章##############
\chapter{第四章}

%############第五章##############
\chapter{第五章}

% 由于结论章节不能出现序号,这边重新定义一个命令
\thesisChapterConclusion
\chapter{}

\section{全文总结}


\section{后续工作展望}

% 参考文献                                    
\thesisbibliography{reference}


% 致谢
%\begin{thesisAcknowledgement}
% 致谢
%\end{thesisAcknowledgement}

% 附录
%\thesisappendix
%\chapter{说明}
%附录是对于一些不宜放在正文中,但有参考价值的内容,可以包括正文内不便列出的冗长公式推导,以备他人阅读方便所需的辅助性数学工具或表格,重复性数据图表,计算程序及说明。(如果没有附录可删除此章,导航窗格中鼠标右键章节标题有直接删除的菜单项) 
%\section{测试}

% 简历
%\begin{thesisResume}
%    \begin{description}[labelsep=0pt, leftmargin=*]
%        \item 姓\hspace{24pt}名:XXX
%        \item 性\hspace{24pt}别:女
%        \item 出生年月:
%    \end{description}
%
%    \section*{\heiti\fontsize{14pt}{16.8pt}\selectfont 学习经历}
%    \begin{description}[labelsep=0pt, leftmargin=*]
%    	\item 研究生:计算机软件与理论,福州大学,2021.09-2024.03
%        \item 学士:计算机软件与理论,福州大学,2017.09-2021.06
%    \end{description}
%
%    \section*{\heiti\fontsize{14pt}{16.8pt}\selectfont 获奖情况}
%    \begin{description}[labelsep=0pt, leftmargin=*]
%        \item 2023年xxx奖学金
%    \end{description}
%\end{thesisResume}

% 在学期间的研究成果及发表的学术论文
%\begin{thesisAccomplish}
%    \section*{\heiti\fontsize{14pt}{16.8pt}\selectfont 在学期间发表或在审论文:} 
%    \hspace{-26pt} 第一作者 (1篇)
%    \begin{enumerate}[labelindent=0pt, leftmargin=*]
%        \item // 
    
%    \end{enumerate}
%    共同作者(1篇)
%    \begin{enumerate}[labelindent=0pt, leftmargin=*]
%        \item XXXXX[C]//Chinese Conference on Pattern Recognition and Computer Vision, 2021.
%    \end{enumerate}

%    \section*{\heiti\fontsize{14pt}{16.8pt}\selectfont 在学期间参与的科研项目:} 
%        \begin{enumerate}[labelindent=0pt, leftmargin=*]
%		\item xxx项目(基金号),2023.01-2026.12,参与
%    \end{enumerate}
%\end{thesisAccomplish}

\end{document}
